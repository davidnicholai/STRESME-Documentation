\chapter{Research Description}
\label{sec:researchdesc}

This chapter discusses the overview of the current state of technology, research objectives, scope and limitations of the research, and significance of the research.

\section{Overview of the Current State of Technology}
\label{sec:overview}

Hearing-impairment is the inability to hear, and it affects one's development when learning how to speak \cite{lasak:2014:HL}. Some people are born with hearing impairment while others may obtain it due to age, illness, trauma, and other factors. Dr. Matkin stated that hearing loss has a grave impact on a child's early development on his or her language, cognition, and social-emotional competence \citeyear{matkin:1999:cechl}. The impact of hearing loss is correlated with the level of hearing loss. There are generally four levels of hearing loss--mild, moderate, severe, and profound \cite{dh:1964:lohl}. A child with hearing loss, especially a profoundly deaf child, one who cannot hear at all, may find it immensely difficult to learn speech and language since they cannot imitate any sounds without auditory feedback \cite{bernstein:1988:STA}. This leads to children not being able to use their vocal chords even though they are  properly functioning. Therefore, speech training systems are developed in order for them to use their vocal chords despite having the inability to hear.

Currently, one way for a hearing-impaired person to learn how to speak is by planting an electronic device called a cochlear implant. It is designed to function as a cochlea, a part of the ear which receives sounds in the form of vibrations and these vibrations are sensed by hair cells. People with hearing-impairment have damage hair cells or less hair cells which makes the cochlea unable to detect the vibrations as it has been received. Cochlear implants stimulate the cochlear nerves without requiring to pass through the hair cells. A person with a cochlear implant may still need to undergo training and therapy in order to adjust to their hearing \cite{blume:2009:AE}. Once a person is able to hear effectively, it can lead to speech training and therapy in order for one to respond accordingly to different sounds. However, a cochlear implant is very costly, ranging from a minimum price of P663,742.50 (\$15,000) to a maximum price of P884,985.51 (\$20,000) \cite{garcia:2015:dtfd}. Thus another solution was implemented to provide an affordable and economical alternative---the speech training system. 

% COMMENT
\begin{comment}
Throughout the years, man has tried to find ways to help the hearing-impaired learn how to speak properly \cite{oyer:1976:CHH}. Oyer \citeyear{oyer:1976:CHH} states that there was a time when hearing-impared were considered unfit to hold citizenship. James Pickett, a professor of speech communication research says "I believe that large improvements in the lives of deaf persons depends on making large improvements in their speech communication" \cite{connor:1971:SDC}. Pickett also states that in the late 19th century, research began for helping the Deaf communicate \cite{connor:1971:SDC}.

One problem of a hearing-impaired student is the lack of access to speech training aids outside of therapy \cite{bernstein:1988:STA}. Extensive practice is required for the student to progress \cite{bernstein:1988:STA}.
\end{comment}
% END OF COMMENT

Speech training systems started during the 19th century. Early speech training systems were implemented using a lip reading technique, wherein the systems take the user's lip movement as input \cite{heracleous:2010:CSA}. The input will then be processed and evaluated by the system with its corresponding phoneme's lip movement. However, numerous phonemes share similar facial and lip shape (visemes) with other phonemes as well; resulting to phonemes being not distinguishable through visual data alone \cite{heracleous:2010:CSA}.
Bernstein et al. \citeyear{bernstein:1988:STA} mentions that visual information is commonly practiced to aid in speech perception. Several visual displays were designed to grasp the attention of children by gamifying the learning process. One is a 'ball game' wherein the ball changes in size depending on the loudness of the sound the child makes. The ball also can be controlled going to the hoop using the child's voice pitch \cite{bernstein:1988:STA}. Another game is a vertical spectrum, which is visually displayed as a changing two-dimensional shape wherein frequency was displayed on the y-axis, and amplitude---the degree of change---on the x-axis \cite{bernstein:1988:STA}. The problem with these games is that they leave the child guessing on whether they are actually doing it right or wrong. There is no specific feedback on how to produce the right sound correctly.

Based on a book written by Susan M. Brookhart \citeyear{brookhart:2008:gef}, feedback varies in certain strategies and these strategies varies in how it is presented. Computer-based speech therapy systems that have been implemented are mostly product-oriented, providing an analysis of the input of the user and the final result. These systems' feedback strategy is modal, specifically visual strategy; however, Brookhart also mentioned that a good feedback in this kind of strategy is a demonstration of how the task should be done \cite{brookhart:2008:gef}. The method does not provide the information of how the sound should be articulated. Furthermore, it does not give comparison to the target or to the previous input of the student which makes it in contrast to giving good feedback in terms of content \cite{brookhart:2008:gef}. The shortcoming of these systems are brought upon by giving visual representations in real time (e.g. game-like visualizations and speech pictures) \cite{oster:2006:cbs}.

Moreover, in training hearing-impaired children, the most preferred feedback is through visual modality while being interactive with the child using it and also providing demonstration on how it should be articulated \cite{brookhart:2008:gef}. This allows the children see the motor gestures that were non-visible, assisting them in developing their speech gestures. The problem that this method posed was that it was seldom evaluated in a pedagogical programme. Often, the visual aids presented was difficult to understand, unnatural, delayed, unattractive, and had no motivational impact on the children \cite{oster:2006:cbs}.

\section{Research Objectives}
\label{sec:researchobjectives}

\subsection{General Objective}
\label{sec:generalobjective}
To implement a feedback system for speech training that provides measurable visual feedback.

\subsection{Specific Objectives}
\label{sec:specificobjectives}

\begin{enumerate}
\item To understand how feedback mechanisms of existing speech training software can be used to improve the implemented system
\item To design a system that trains hearing-impaired children in uttering phonemes through visual feedback
\item To implement a system that recognizes different phonetic sounds from an audio clip that utters a phoneme sound
\item To identify and compare different methods and design of visual feedback in speech training systems
\item To evaluate the accuracy of the implemented system's model
\item To evaluate the effectiveness of the implemented speech feedback system
\end{enumerate}

\section{Scope and Limitations of the Research}
\label{sec:scopelimitations}

The research will strictly focus on feedback mechanisms. Motivating or having the users be engaged into using the system will not be considered as the research will be focusing on providing visual feedback.

The research will cater to only having a fixed set of words in the system. The set of words is provided by the institute the researchers have partnered with. Each word in the set will then be syllabicated into phonemes. Only the syllabicated words will be provided with the measurable visual feedback. The visual feedback for the entire word may only display "correct" or "incorrect".

The feedback system will only process WAV audio clips with the user saying phonemes as input into the feedback system. Components of the GUI such as the video playback and web camera display will only be used as additional aid to the user. The feedback to be displayed to the trainee will only be visual, meaning the feedback will not be discerned by senses other than sight.

For the implementation of the feedback system, the master's thesis of Natalie Agustin \citeyear{agustin:2014:SOM} will be used. Agustin's research is a mapping prototype using Self-Organizing Maps (SOMs) which encodes a voice recording into coefficients and distinguishes which vowel a user had spoken. Agustin's thesis will be used for the system to distinguish which phoneme the trainee had spoken. The speech feedback system will cater to a fixed set of phonemes. The speech feedback system will run on a desktop or a laptop machine, as these are accessible and widely-used. 

%SCOPE OF PHONEMES WE WILL CONSIDER

\section{Significance of the Research}
\label{sec:significance}

The thesis will assist the Deaf in learning the pronunciations of different phonemes. Current speech training software in the industry only show graphical illustrations of the user's voice input and the goal in which the user must achieve. This method however lacks in providing measurable feedback of the input of the user. The thesis will be providing feedback as to how the trainee can correctly pronounce phonemes by displaying how far or how near they are from the correct pronunciation. The visual feedback provided to the users of the system will improve their performance in learning different phonemes.

Deaf schools may make use of the system as a teaching and evaluation tool for their students.

The thesis may serve as a basis for future implementations of speech training software which may cater not only to a fixed set of phonemes, but with a bigger set of training data.

\section{Research Methodology}
\label{sec:methodology}

\subsection{Research Activities}
The research activities will be divided into the following phases:
\begin{enumerate}
\item Gathering Literature
\item Research and Comparison
\item Partnering with an Institute
\item Data Collection and Model Training
\item Implementation of the Software
\item Testing the Accuracy of the Model
\item Field Tests and Evaluation of the Feedback System
\end{enumerate}

\subsubsection{Gathering Literature}
The appropriate literature---visual feedback mechanisms, speech training systems, phonemes, speech training---are gathered in order to design and implement an effective speech feedback system. Several textbooks and journals are available in the institution where the proponents are studying in.

\subsubsection{Research and Comparison}
After having gathered data, the group is then tasked to research and compare different literature and ideas. There are several ways of providing visual and measurable feedback to users. However, the group should only select the more effective ways of providing said kind of feedback. With this, it is important to look behind the science of these visual feedback and how effective they are.

There are several speech training software created by different companies. IBM, for one, created SpeechViewer which is used by speech therapists to train those with hearing-impairment. The software provides a feature wherein the therapist can record training sessions in order to see the improvements of their patients. Though the software may be outdated, it may serve as a reference on how a speech training process works. Viewing other speech training software may help the thesis group in gathering ideas in order to create an effective feedback system.

\subsubsection{Partnering with an Institute}
The researchers searched for Deaf institutes who are open in allowing the researchers to conduct interviews with their teachers and students for the sole purpose of creating an effective speech feedback system. The Philippine Institute for the Deaf has agreed to aid in this research. The researchers inquired with the institute in aiding them with this research, however, since it was only an inquiry, the institute did not allow them to document the session as there was no agreement or contract yet between the two parties.

\subsubsection{Implementation of the Software}
The feedback system must be effective and easy to use. The mapping software of Natalie Agustin \citeyear{agustin:2014:SOM} will be integrated into the speech feedback system in order for the voice input, which the proponents have implemented, to be recognized and provide the corresponding mapping which determines the phoneme that was spoken by the user. Implementation will be performed until the second stage of the thesis.

\subsubsection{Data Collection and Model Training}
The researchers will collect audio clips of teachers pronouncing the different words and syllabicated words. Each word and phoneme will be recorded 10 times in order to train the SOM model and increase its accuracy. Once data have been gathered, the audio clips will then be framed, cutting out noise in the audio. The SOM will then be trained by feeding all processed audio recordings into the SOM. The trained SOM should then be able to recognize phonemes.

\subsubsection{Testing the Accuracy of the Model}
The accuracy of the SOM model will be determined in order to see if the model classifies processed audio clips correctly. The accuracy is computed by the number of correct classifications over the total number of instances. This will be achieved by having the researchers record their own utterances to see if the SOM will map each sound to its corresponding node. The built SOM model will be used to classify the phonemes pronounced in the testing dataset.

\subsubsection{Field Tests and Evaluation of the Feedback System}
To determine how effective the implemented software is, the researchers are going to have the program tested by the hearing-impaired, guided by their speech trainers. The researchers will have to go to the school and teach the teachers on how to use the system. The researchers would like to know if the visual feedback given performs well in guiding the student in learning the fixed set of phonemes. This way, the researchers will know how to further improve the software. The researchers will be providing the teachers with surveys in order to determine how the feedback mechanism had helped the students in learning.

The field test will be conducted by having two groups, the control group and the experimental group. Each group will consist of 5 students. The control group will make use of their current method of teaching, and the experimental group will make use of the feedback software. Each group will not be taught altogether, but rather each student will have a session to use the software (if he or she is in the experimental group) or he or she will have a one-on-one session with the teacher (if he or she is in the control group). In each session of each student, the teachers will be tracking their learning curve by noting how fast a child learned a phoneme. A questionnaire or evaluation sheet will be provided to teachers in order to track each student's sessions. The field test will be conducted for four weeks.

The performance metric for system evaluation will be each student's learning curve. Based from the feedback the student is getting, with each try, the researchers will see if the student is getting closer to the target.

\subsubsection{Resources}
To perform the field tests, the Philippine Institute for the Deaf has agreed to provide the researchers with ten students ranging from the fourth elementary grade up to the sixth elementary grade. A fixed fee of PHP 300.00 per researcher is needed before the field test can be performed. Having four researchers in this study, the total fee will amount to PHP 1200.00. A laptop machine with an integrated microphone is also needed in order to deploy the speech feedback system to the institute.

\pagebreak
\subsection{Calendar of Activities}

% Table \ref{tab:timetableactivities} shows a Gantt chart of the activities. Each bullet represents approximately one week worth of activity.
The tables below each show a Gantt chart of activities. Each bullet represents approximately one week's worth of activity.

%
%  the following commands will be used for filling up the bullets in the Gantt chart
%
\newcommand{\weekone}{\textbullet}
\newcommand{\weektwo}{\textbullet \textbullet}
\newcommand{\weekthree}{\textbullet \textbullet \textbullet}
\newcommand{\weekfour}{\textbullet \textbullet \textbullet \textbullet}

\begin{table}[!htbp]
\centering
\caption{Timetable of Activities for March 2015 to July 2015} \vspace{0.25em}
\begin{tabular}{|p{2in}|c|c|c|c|c|} \hline
\centering Activities			& Mar		& Apr			& May			& Jun		& Jul			\\ \hline
Gathering Literature			& \weekfour	&				&				&~~~~~~		&					\\ \hline
Research and Comparison			&			&\weektwo~~~~	&~\weekthree	&			&				\\ \hline
Partnering with an Institute	&			&				&				&			&~\weekone~~~	\\ \hline
Documentation					& \weekfour	&\weektwo~~~~	&~\weekthree	&			&~\weekone~~~	\\ \hline
\end{tabular}
\label{tab:timetableactivities1}
\end{table}

\begin{table}[!htbp]
\centering
\caption{Timetable of Activities for August 2015 to December 2015} \vspace{0.25em}
\begin{tabular}{|p{2in}|c|c|c|c|c|} \hline
\centering Activities				& Aug		& Sept		& Oct			& Nov		& Dec			\\ \hline
Partnering with an Institute		& 			&			&				&\weekone~~~&				\\ \hline
Implementation of the Software		& \weekfour	& \weekfour	& \weekfour		& \weekfour	&\weektwo~~~	\\ \hline
Data Collection and Model Training	&			&			&				&			&\weektwo~~~	\\ \hline
Documentation						& \weekfour	& \weekfour	& \weekfour		& \weekfour	&\weektwo~~~	\\ \hline
\end{tabular}
\label{tab:timetableactivities2}
\end{table}

\begin{table}[!htbp]
\centering
\caption{Timetable of Activities for January 2016 to May 2016} \vspace{0.25em}
\begin{tabular}{|p{2in}|c|c|c|c|c|} \hline
\centering Activities						& Jan		& Feb		& Mar		& Apr		& May		\\ \hline
Field Tests and Evaluation of the System	&~\weekthree&\weektwo~~~&\weekfour	&~~~~~~		&~~~~~		\\ \hline
Documentation								&~\weekthree&\weektwo~~~&\weekfour	&			&			\\ \hline
\end{tabular}
\label{tab:timetableactivities3}
\end{table}

\begin{table}[!htbp]
\centering
\caption{Timetable of Activities for June 2016 to August 2016} \vspace{0.25em}
\begin{tabular}{|p{2in}|c|c|c|c|c|} \hline
\centering Activities						& Jun		& Jul		& Aug		& 			& 		\\ \hline
Field Tests and Evaluation of the System	&~~\weektwo	&			&			&~~~~~~		&~~~~~		\\ \hline
Documentation								&~~\weektwo	&\weekfour	&			&			&			\\ \hline
\end{tabular}
\label{tab:timetableactivities4}
\end{table}