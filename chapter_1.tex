\chapter{Research Description}
\label{sec:researchdesc}

This chapter discusses the overview of the current state of technology, research objectives, scope and limitations of the research, and significance of the research.

\section{Overview of the Current State of Technology}
\label{sec:overview}

Hearing impairment is the inability to hear, and it may affect one’s development when learning how to speak \cite{lasak:2014:HL}. Some people are born with hearing impairment, and some obtain it due to age, illness, trauma, and other factors. An extremely deaf child may find it immensely difficult to learn speech and language since they cannot imitate any sounds without auditory feedback \cite{bernstein:1988:STA}. This leads to children not being able to use their vocal chords even though it is properly functioning.

\begin{comment}
Throughout the years, man has tried to find ways to help the hearing-handicapped learn how to speak properly \cite{oyer:1976:CHH}. Oyer \citeyear{oyer:1976:CHH} states that there was a time when hearing-handicapped were considered unfit to hold citizenship. James Pickett, a professor of speech communication research says "I believe that large improvements in the lives of deaf persons depends on making large improvements in their speech communication" \cite{connor:1971:SDC}. Pickett also states that in the late 19th century, research began for helping the Deaf communicate \cite{connor:1971:SDC}.

One problem of a hearing-impaired student is the lack of access to speech training aids outside of therapy \cite{bernstein:1988:STA}. Extensive practice is required for the student to progress \cite{bernstein:1988:STA}.
\end{comment}

Early speech training software were implemented using a lip reading technique, wherein the systems take the user’s lip movement as input \cite{heracleous:2010:CSA}. However, many phonemes share similar facial and lip shape (visemes) with other phonemes as well; resulting to phonemes not being discernable through visual data alone \cite{heracleous:2010:CSA}.

Bernstein et al. \citeyear{bernstein:1988:STA} mentions that visual information is commonly practiced to aid in speech perception. Several visual displays were designed to grasp the attention of children wherein they play games to learn. One of the games is a ’ball game’ wherein the ball changes in size depending on the loudness of the sound the child makes. The ball also can be controlled going to the hoop using the child’s voice pitch \cite{bernstein:1988:STA}. Another game is a vertical spectrum, which is visually displayed as a changing 2-dimensional shape wherein frequency was displayed on the y-axis, and amplitude---the degree of change---on the x-axis \cite{bernstein:1988:STA}. The problem with this game is that it only leaves the children to guess on whether he is actually doing right or wrong. There is no specific feedback on how to produce the right sound correctly.

Currently, one way for a hearing-impaired person to learn how to speak is by planting an electronic device called a cochlear implant. It is designed to function as a cochlea, a part of the ear which receives sounds in the form of vibrations and these vibrations are sensed by hair cells. Deaf people are unable to hear because their hair cells cannot detect the vibrations as received by the cochlea due to damaged hairs cells, or they have less hair cells. Cochlear implants stimulate the cochlear nerves without going the hair cells anymore. A person with a cochlear implant may still undergo training and therapy in order to adjust to their hearing \cite{blume:2009:AE}. Once a person hears properly, it can lead to speech training and therapy in order for one to respond accordingly to different sounds.
					
Computer-based speech therapy systems that have been implemented are mostly product-oriented, providing an analysis of the input of the user and the final result. The method, however, does not provide the information of how the sound should be articulated. The shortcoming of these systems are brought upon by giving visual representations in real time i.e. game-like visualizations and speech pictures \cite{oster:2006:cbs}.
									
Moreover, in training hearing-impaired children, the most preferred feedback is through visual modality. This allows the children see the motor gestures that were non-visible, assisting them in developing their speech gestures. The problem that this method posed was that it was seldom evaluated in a pedagogical programme. Often, the visual aids presented was difficult to understand, unnatural, delayed, unattractive, and had no motivational impact on the children \cite{oster:2006:cbs}.
				
\section{Research Objectives}
\label{sec:researchobjectives}

\subsection{General Objective}
\label{sec:generalobjective}

\begin{comment}
To implement a speech training software that provides visual feedback to assist people with hearing impairment in order to learn how to speak vowel sounds.
\end{comment}

To implement a feedback system that provides scalar visual feedback by means of colour representation.

\subsection{Specific Objectives}
\label{sec:specificobjectives}

\begin{enumerate}
\item To find schools for the hearing-impaired who may assist the group with the testing and implementation of the research.
\item To identify and compare different methods and design of visual feedback in speech training systems.
\item To select the optimal colour representation to be used as visual scalar feedback.
\item To select the optimal visual feedback method based on usability and effectiveness
\item To evaluate the accuracy of the implemented speech training software’s visual feedback.
\item To evaluate the effectiveness of the implemented speech training software
\end{enumerate}

\section{Scope and Limitations of the Research}
\label{sec:scopelimitations}

For the implementation of the speech training software, the master's thesis of Natalie Agustin \citeyear{agustin:2014:SOM} will be used. Agustin's program is a mapping software which receives voice input from the user and identifies the vowel sound made by the user and displays the user’s articulation---the movement of the lips, tongue, jaw, and other speech organs. Agustin's thesis will be used for the system to know which vowel the user had spoken.
The speech training software will be written in Java, a cross-platform programming language, making the system usable by different operating systems.
The speech training software will run on a desktop platform, as desktop computers are accessible and widely-used.
The proponents will review only speech training software with free trials.
The feedback system will only cater to vowel sounds, as the mapping software to be used only serves vowel sounds.

\section{Significance of the Research}
\label{sec:significance}

The thesis will assist the Deaf in learning the pronunciations of vowel sounds correctly. Deaf communities may make use of the software, guided by a professional speech trainer, in his or her development with regard to speech training.
The thesis may serve as a basis for future implementations of Deaf speech training software which may cater not only vowels, but also consonants.

The thesis will assist the Deaf in learning the pronunciations of vowel sounds accurately. Current speech training software in the industry only shows a graphical illustration of the user's voice input and the goal the user must achieve. This method however lacks in providing scalar feedback of the input of the user. The thesis will be providing a feedback on what sound is the user currently producing and what he/she needs to do in order to correct himself/herself. Deaf communities may make use of the software, guided by a professional speech trainer, in his or her development with regard to speech training. The thesis may serve as a basis for future implementations of Deaf speech training software which may cater not only vowels, but also consonants.